 % Authors:  Nils Carqueville, Daniel Murfet
 
\documentclass{compositio}
\usepackage{stmaryrd}
\usepackage{amsmath, amscd, amssymb, mathrsfs, accents, amsfonts}
\usepackage{url}
\usepackage[all]{xy}
\usepackage{longtable}
\usepackage{dsfont}
\usepackage{tikz}
\def\nicenocolourscheme{\shadedraw[top color=gray!2, bottom color=gray!25, draw=gray!50!black, dashed]}
\definecolor{Myblue}{rgb}{0,0,0.6}
\usepackage[a4paper,colorlinks,citecolor=Myblue,linkcolor=Myblue,urlcolor=Myblue,pdfpagemode=None]{hyperref}

\SelectTips{cm}{}

\newtheorem{theorem}{Theorem}[section]
\newtheorem{proposition}[theorem]{Proposition}
\newtheorem{lemma}[theorem]{Lemma}
\newtheorem{corollary}[theorem]{Corollary}
\newtheorem*{theoremn}{Theorem}

\theoremstyle{definition}
\newtheorem{definition}[theorem]{Definition}
\newtheorem{example}[theorem]{Example}
\newtheorem{remark}[theorem]{Remark}
\newtheorem{s}[theorem]{}
\newtheorem*{setup}{Setup}
\newtheorem*{propositionn}{Proposition}

\numberwithin{equation}{section}

% Operators
\def\eval{\operatorname{ev}}
\def\coev{\operatorname{coev}}
\def\res{\operatorname{Res}}
\def\sg{\operatorname{sg}}
\def\Inj{\operatorname{Inj}}
\def\inc{\operatorname{inc}}
\def\Proj{\operatorname{Proj}}
\def\Coker{\operatorname{Coker}}
\def\Ker{\operatorname{Ker}}
\def\Im{\operatorname{Im}}
\def\free{\operatorname{free}}
\def\can{\operatorname{can}}
\def\ac{\operatorname{ac}}
\def\HH{\operatorname{HH}}
\def\K{\mathbf{K}}
\def\D{\mathbf{D}}
\def\N{\mathbf{N}}
\def\sing{\operatorname{Sg}}
\def\Hom{\operatorname{Hom}}
\def\uHom{\underline{\Hom}}
\def\modd{\operatorname{mod}}
\def\Modd{\operatorname{Mod}}
\def\Grmodd{\operatorname{GrMod}}
\def\CM{\operatorname{CM}}
\def\Ker{\operatorname{Ker}}
\def\Spec{\operatorname{Spec}}
\def\straightK{\operatorname{K}}
\def\straightC{\operatorname{C}}
\def\holim{\operatorname{hocolim}}
\DeclareMathOperator{\Ext}{Ext}
\DeclareMathOperator{\coh}{coh}
\DeclareMathOperator{\serre}{S}
\DeclareMathOperator{\Flat}{Flat}
\DeclareMathOperator{\qc}{qc}
\DeclareMathOperator{\Perf}{Perf}
\DeclareMathOperator{\Map}{Map}
\DeclareMathOperator{\Qco}{Qco}
\DeclareMathOperator{\Tr}{Tr}
\DeclareMathOperator{\End}{End}
\DeclareMathOperator{\rank}{rank}
\DeclareMathOperator{\tot}{Tot}
\DeclareMathOperator{\skos}{K}
\DeclareMathOperator{\hht}{ht}
\DeclareMathOperator{\depth}{depth}
\DeclareMathOperator{\STr}{STr}
\DeclareMathOperator{\tr}{tr}
\DeclareMathOperator{\ch}{ch}
\DeclareMathOperator{\str}{str}
\DeclareMathOperator{\hmf}{hmf}
\DeclareMathOperator{\HMF}{HMF}
\DeclareMathOperator{\HF}{HF}
\DeclareMathOperator{\pr}{pr}
\DeclareMathOperator{\At}{At}
\DeclareMathOperator{\mff}{mf}
\DeclareMathOperator{\MF}{MF}
\DeclareMathOperator{\Sh}{Sh}

\begin{document}

% Commands
\def\Res{\res\!}
\newcommand{\cat}[1]{\mathcal{#1}}
\newcommand{\lto}{\longrightarrow}
\newcommand{\xlto}[1]{\stackrel{#1}\lto}
\newcommand{\mf}[1]{\mathfrak{#1}}
\newcommand{\md}[1]{\mathscr{#1}}
\newcommand{\intvar}{\bs{x}_{\textup{int}}}
\newcommand{\extvar}{\bs{x}_{\textup{ext}}}
\newcommand{\qderu}[2]{\mathbf{D}^{#1}(#2)}
\newcommand{\ud}{\mathrm{d}}
\def\l{\,|\,}
\def\cf{\boldsymbol{cf}}
\def\bx{\boldsymbol{x}}
\def\by{\boldsymbol{y}}
\def\ba{\boldsymbol{a}}
\def\bb{\boldsymbol{b}}
\def\totimes{\otimes}
\def\di{Q}
\newcommand{\cotimes}[1]{\,\widehat{\otimes}_{#1}\,}
\def\QQ{\mathds{Q}}
\def\krc{C}
\def\diffm{d}
\def\diffh{d_{\chi}}
\def\redh{\overline{H}}
\def\ZZ{\mathds{Z}}
\def\bs{\boldsymbol}
\def\Ztwo{\mathds{Z}_2}
\def\mdual{^{\vee}}
\def\KR{\operatorname{KR}}
\def\I{\!\operatorname{i}\!}
\def\E{\operatorname{e}\!}
\def\sln{\mathfrak{sl}(N)}
\def\nN{\mathds{N}}
\def\nZ{\mathds{Z}}
\def\nQ{\mathds{Q}}
\def\nR{\mathds{R}}
\def\nC{\mathds{C}}
\def\Bar{\mathds{B}}
\def\cBar{\widehat{\mathds{B}}}
\def\Ae{A^{\operatorname{e}}}
\newcommand{\be}{\begin{equation}}
\newcommand{\ee}{\end{equation}}
\def\Xcirc{%
\begin{tikzpicture}[inner sep=0mm]
\node (X) at (0,0) {$X$};
\node (0) at (0,0) [circle,inner sep=0.99pt, thin,draw=black,fill= white] {};
\end{tikzpicture}%
}
\def\Xbul{%
\begin{tikzpicture}[inner sep=0mm]
\node (X) at (0,0) {$X$};
\node (0) at (0,0) [circle,inner sep=0.99pt, thin,draw=black,fill= black] {};
\end{tikzpicture}%
}

\usetikzlibrary{arrows,calc,decorations.pathreplacing,decorations.markings,shapes.geometric,shadows}
\tikzset{
    string/.style={draw=#1, postaction={decorate}, decoration={markings,mark=at position .51 with {\arrow[draw=#1]{>}}}},
    costring/.style={draw=#1, postaction={decorate}, decoration={markings,mark=at position .51 with {\arrow[draw=#1]{<}}}},
    ostring/.style={draw=#1, postaction={decorate}, decoration={markings,mark=at position .47 with {\arrow[draw=#1]{>}}}},
    ustring/.style={draw=#1, postaction={decorate}, decoration={markings,mark=at position .56 with {\arrow[draw=#1]{>}}}},
    oostring/.style={draw=#1, postaction={decorate}, decoration={markings,mark=at position .43 with {\arrow[draw=#1]{>}}}},
    uustring/.style={draw=#1, postaction={decorate}, decoration={markings,mark=at position .59 with {\arrow[draw=#1]{>}}}},
    directed/.style={string=blue!50!black}, 
    odirected/.style={ostring=blue!50!black}, 
    udirected/.style={ustring=blue!50!black}, 
    oodirected/.style={oostring=blue!50!black}, 
    uudirected/.style={uustring=blue!50!black},     
    redirected/.style={costring= blue!50!black},
}





\title{Rise of the Planet of the Coevaluations}
\author{Nils Carqueville}
\email{nils.carqueville@physik.uni-muenchen.de}
\address{Arnold Sommerfeld Center for Theoretical Physics, LMU M\"unchen \& Excellence Cluster Universe}

\author{Daniel Murfet}
\email{daniel.murfet@math.ucla.edu}
\address{Department of Mathematics, UCLA}

\classification{TODO}

\begin{abstract}
They take over. 
\end{abstract}

\maketitle


\section{Zorro moves}\label{sec:Zorro}

In this section we will show that the bicategory $\mathcal{LG}$ of Landau-Ginzburg models has dualities [TODO: come up with better nomenclature]. Let us fix two arbitrary potentials $W\in A:= k[x_1,\ldots,x_n]\equiv k[x]$ and $V\in B:= k[z_1,\ldots,z_m]\equiv k[z]$. Then we want to prove that for any matrix factorisation $X\in \hmf(B \otimes_k A, V-W)$ and its dual $X^* = X^\vee[n]\in \hmf(A \otimes_k B, W-V)$ the Zorro moves~\eqref{TODO} are satisfied. 

Let us consider the identity~\eqref{TODO} in more detail: 
\be\label{Zorro1detail}
\begin{tikzpicture}[very thick,scale=1.0,color=blue!50!black, baseline=0cm]

\fill (0.2,1.6) circle (0pt) node {{\small $\cBar$}};
\fill (-0.2,-1.6) circle (0pt) node {{\small $\cBar$}};

\fill (1,1.8) circle (2.5pt) node[right] {{\small $\lambda_X$}};
\fill (-1,-1.8) circle (2.5pt) node[left] {{\small $\rho_X^{-1}$}};

\fill (-1.25,-2.25) circle (0pt) node {{\footnotesize $z$}};
\fill (-0.75,-2.25) circle (0pt) node {{\footnotesize $x$}};

\fill (-1.5,0) circle (0pt) node {{\footnotesize $z\vphantom{z'}$}};
\fill (-0.5,0) circle (0pt) node {{\footnotesize $x'$}};
\fill (0.5,0) circle (0pt) node {{\footnotesize $z'$}};
\fill (1.5,0) circle (0pt) node {{\footnotesize $y\vphantom{z'}$}};

\fill (1.25,2.25) circle (0pt) node {{\footnotesize $y$}};
\fill (0.75,2.25) circle (0pt) node {{\footnotesize $z\vphantom{y}$}};

\draw[dashed] (-0.5,0.75) .. controls +(0,0.75) and +(-0.25,-0.75) .. (1,1.8);
\draw[dashed] (0.5,-0.75) .. controls +(0,-0.75) and +(0.25,0.75) .. (-1,-1.8);

\draw[line width=0] 
(1,2.7) node[line width=0pt] (A) {{\small $X$}}
(-1,-2.7) node[line width=0pt] (A2) {{\small $X$}}; 
\draw[redirected] (0,0) .. controls +(0,1) and +(0,1) .. (-1,0);
\draw[redirected] (1,0) .. controls +(0,-1) and +(0,-1) .. (0,0);
\draw (-1,0) -- (A2); 
\draw (1,0) -- (A); 
\end{tikzpicture}
=
\begin{tikzpicture}[very thick,scale=1.0,color=blue!50!black, baseline=0cm]
\draw[line width=0] 
(0,2.7) node[line width=0pt] (A) {{\small $X$}}
(0,-2.7) node[line width=0pt] (A2) {{\small $X$}}; 
\draw (A2) -- (A); 
\end{tikzpicture}
\ee
Here we indicated our choices for variable names in the various domains, as well as the fact that we use the completed bar complex~$\cBar$ as a model for the unit endomorphism of the 0-cell~$W$. Writing out the coevaluation~\eqref{TODO} the left-hand side of~\eqref{Zorro1detail} becomes 
\be\label{Zorro1superdetail}
Z := 
\begin{tikzpicture}[very thick,scale=0.8,color=blue!50!black, baseline=1.7cm,line/.style={&gt;=latex}]

\draw (-1.3,0) -- (-1.3,4.8); 
\draw (-4.9,-3) -- (-4.9,4.8); 
\draw[redirected] (-1.3,4.8) .. controls +(0,1.75) and +(0,1.75) .. (-4.9,4.8);

\draw (0,0) -- (0,8.5); 

\draw[dashed] (1.3,0.4) -- (1.3,4.5);

\draw[dashed] (-2.6,0.4) -- (-2.6,1.1);
\draw[dashed] (-2.6,1.1) .. controls +(1,0.7) and +(-1,-0.7) .. (2.6,1.9);
\draw[dashed] (2.6,1.9) -- (2.6,3);

\draw[line width=0.75pt] 
(0,4.5) node[fill=white,draw,text width=5cm,align=center]{{\footnotesize$1_{X^*} \otimes 1_X \otimes (\varepsilon\circ\Psi)\vphantom{1_{\cBar} \otimes \sum_{b\geq 0} (-1)^b \At(d_{X^*\otimes X})^b(\iota)}$}} 
(0,3) node[fill=white,draw,text width=5cm,align=center]{{\footnotesize$1_{X^*} \otimes 1_X \otimes \times\vphantom{1_{\cBar} \otimes \sum_{b\geq 0} (-1)^b \At(d_{X^*\otimes X})^b(\iota)}$}}
(0,1.5) node[draw,text width=5cm,align=center]{{\footnotesize$\vphantom{1_{\cBar} \otimes \sum_{b\geq 0} (-1)^b \At(d_{X^*\otimes X})^b(\iota)}$}}
(0,0) node[fill=white,draw,text width=5cm,align=center]{{\footnotesize$1_{\cBar} \otimes \sum_{b\geq 0} (-1)^b \At(d_{X^*\otimes X})^b(\iota)$}};

\fill (3.2,4.5) circle (0pt) node[right] {{\small $\gamma_4$}};
\fill (3.2,3) circle (0pt) node[right] {{\small $\gamma_3$}};
\fill (3.2,1.5) circle (0pt) node[right] {{\small $\gamma_2$}};
\fill (3.2,0) circle (0pt) node[right] {{\small $\gamma_1$}};

\fill (0,7.8) circle (2.5pt) node[right] {{\small $\lambda_X$}};
\fill (-4.9,-2.2) circle (2.5pt) node[left] {{\small $\rho_X^{-1}$}};

\draw[dashed] (-4.9,-2.2) .. controls +(0.25,1) and +(0,-1) .. (-2.6,-0.4);
\draw[dashed] (-3.1,6) .. controls +(0,1) and +(-0.25,-1) .. (0,7.8);

\end{tikzpicture}
\ee
where $\iota = \sum_j (-1)^{|e_j|} e_j^* \otimes e_j$ for an $A$-basis $\{e_j\}$ of~$X$ in~$\gamma_1$, and the twist map~$\gamma_2$ produces Koszul signs coming from commuting non-commutative forms from the left to the very right. 

To prove that~\eqref{Zorro1superdetail} is indeed homotopic to the identity we first concentrate on the map 
$$
\Gamma := \gamma_3 \circ \gamma_2 \circ \gamma_1 \circ \rho_X^{-1}: X \longrightarrow X \otimes_A X^* \otimes _B X \otimes_{\Ae} \cBar \, . 
$$
If we understand the bar complex as $\Omega_A(A\otimes_k B)$ then as explained in section~\ref{TODO} only $d_{X^*}$ contributes to the Atiyah class in~$\gamma_1$. As a result~$\Gamma$ is basically the shuffle product of two telescopic series of the Atiyah classes for~$X$ and $X^*$. This can be expressed in terms of the Atiyah class of the tensor product $X^* \otimes X$: 

\begin{lemma}
$\Gamma(e_q) = \sum_{n\geq 0} \sum_j (-1)^{|e_j| + n} (\At(d_{X\otimes X^*}))^n(e_q \otimes e_j^* \otimes e_j)$. 
\end{lemma}

\begin{proof}
We compute $\Gamma(e_q)$, using $\At(d_X)(e_q) = (-1)^{|e_q|+1} e_k \otimes d(d_X)_{kq}$ and paying attention to Koszul signs: 
\begin{align}
e_q & \stackrel{\rho_X^{-1}}{\longmapsto} \sum_{a\geq 0} (-1)^a \At(d_X)^a(e_q) \nonumber \\
& \qquad\; = \sum_{a\geq 0} (-1)^{a + (|e_q|+1)+\ldots+(|e_q|+a)} e_{k_a} \otimes d(d_X)_{k_a k_{a-1}} \ldots d(d_X)_{k_1 q} \nonumber \\
& \stackrel{\gamma_1}{\longmapsto} \Big( \sum_{a\geq 0} (-1)^{a + a|e_q| + {a+1\choose 2}} e_{k_a} \otimes d(d_X)_{k_a k_{a-1}} \ldots d(d_X)_{k_1 q} \Big) \nonumber \\
& \qquad\; \otimes \Big( \sum_{b\geq 0} \sum_j (-1)^{|e_j|+b} (-1)^{(|e_j|+1)+\ldots+(|e_j|+b) + b|e_j|} e^*_{l_b} \otimes e_j \otimes d(d_{X^*})_{l_b l_{b-1}} \ldots d(d_{X^*})_{l_1 j} \Big) \nonumber \\
& \stackrel{\gamma_3 \circ \gamma_2}{\longmapsto} \sum_{n\geq 0} \sum_j (-1)^{|e_j|+n} \sum_{a=0}^n(-1)^{a|e_q| + {a+1\choose 2} + {n-a+1\choose 2}} \sum_{\sigma \in \Sh(n-a,a)} (-1)^{|\sigma|} e_{k_a} \otimes e^*_{l_{n-a}} \otimes e_j \nonumber \\
& \qquad\;   \otimes \sigma_\bullet \left( d(d_{X^*})_{l_{n-a} l_{n-a-1}} \ldots d(d_{X^*})_{l_1 j} d(d_X)_{k_a k_{a-1}} \ldots d(d_X)_{k_1 q} \right) \nonumber \\
& \qquad\; =  \sum_{n\geq 0} \sum_j (-1)^{|e_j|+n} \sum_{a=0}^n (-1)^{a|e_q| + {a+1\choose 2} + {n-a+1\choose 2} + a(n-a)} \sum_{\sigma \in \Sh(a,n-a)} (-1)^{|\sigma|}  \label{31} \\
& \qquad\qquad \cdot e_{k_a} \otimes e^*_{l_{n-a}} \otimes e_j \otimes \sigma_\bullet \left( d(d_X)_{k_a k_{a-1}} \ldots d(d_X)_{k_1 q} d(d_{X^*})_{l_{n-a} l_{n-a-1}} \ldots d(d_{X^*})_{l_1 j} \right) \nonumber \\
& \qquad\; =  \sum_{n\geq 0} \sum_j (-1)^{|e_j|+n} \big( \At(d_X) + \At(d_{X^*}) \big)^n (e_q \otimes e^*_j \otimes e_j) \label{32} \\
& \qquad\; =  \sum_{n\geq 0} \sum_j (-1)^{|e_j|+n} \At(d_{X\otimes X^*} \big)^n (e_q \otimes e^*_j \otimes e_j) \, . \nonumber
\end{align}
To understand the penultimate step we note that the sign $(-1)^{a|e_q| + {a+1\choose 2} + {n-a+1\choose 2} + a(n-a)}$ with $\sigma=\operatorname{id}$ in~\eqref{31} is precisely that of the contribution to~\eqref{32} where $\At(d_{X^*})$ first acts $n-a$ times on $e_q\otimes e^*_j\otimes e_j$, followed by $\At(d_X)^a$. The sign $(-1)^{|\sigma|}$ appears in~\eqref{32} if some $\At(d_X)$ acts before some of the $\At(d_{X^*})$. 
\end{proof}

Since the Zorro map~$Z$ is $k[z,x]$-linear it is fixed by its action on basis elements~$e_q$, so we find 
\be\label{Zorrointermediate}
Z = \sum_j (-1)^{|e_j| + n + {n+1\choose 2}} \Res_{k[x']} \left[ \frac{\str (\underline\lambda' (\varepsilon\circ\Psi) \At(d_{X\otimes X^*})^n (-\otimes e_j^* \otimes e_j) ) \underline{dx'}}{\partial_{x'_1}W \ldots \partial_{x'_n} W} \right]
\ee
where $\underline{dx'}=dx'_1\ldots dx'_n$ and $\underline\lambda'=\lambda'_1\ldots \lambda'_n$ with $\lambda'_i=\partial_{x'_i}d_X(z,x')$. To arrive at the above expression for~$Z$ we also used that by naturality we have $(\widetilde\eval_X \otimes 1_X)\circ \gamma_4 = (1_{\cBar}\otimes 1_X \otimes(\varepsilon \Psi))\circ (\widetilde\eval_X \otimes 1_X \otimes 1_{\cBar})$. 

The Zorro map~\eqref{Zorrointermediate} will generally be the identity on~$X$ up to homotopy, but determining this homotopy directly is not easy. However, we can and will make use of the fact that the Kapustin-Li trace~\eqref{TODO} is non-degenerate, so~$Z$ equals~$1_X$ up to homotopy if
\be\label{ZorroinKL}
\langle Z \psi' \rangle_X = \langle \psi' \rangle_X
\ee
for all closed endomorphisms~$\psi'$ of~$X$. If we write $\psi:=\psi'\underline\lambda\underline\mu$ with $\lambda_i=\partial_{x_i}d_X(z,x)$ and $\mu_i=\partial_{z_i}d_X(z,x)$ then~\eqref{ZorroinKL} holds if
$$
\str(Z\psi) = \str(\psi) \quad \mod \partial_{y_i} W, \partial_{z_i} V \, .
$$
We will denote equality modulo the derivatives $f_i(y):= \partial_{y_i}W$ and $g_i(z):= \partial_{z_i}V$ by ``$\equiv$'', and we write $\langle\!\langle-\rangle\!\rangle$ for the map $\cBar\longrightarrow k[z,y]$ which acts as
$$
\beta \longmapsto (-1)^{n+1\choose 2} \Res_{k_[x']} \left[ \frac{(\varepsilon\circ\Psi)(\beta) \underline{dx'}}{\partial_{x'_1}W \ldots \partial_{x'_n} W} \right] .
$$
Then we can compute 
\begin{align}
\str(Z\psi) & = \sum_i (-1)^{|e_i|} e^*_i \big( Z(\psi(e_i)) \big) \nonumber \\
& = \sum_{i,j} (-1)^{|e_i| + |e_j| + n} e^*_i \Big( \big\langle\!\big\langle \str( \underline\lambda' \At(d_{X\otimes X^*})^n (\psi(e_i)\otimes e^*_j \otimes e_j)) \big\rangle\!\big\rangle \Big) \nonumber \\
& = \sum_{i,j} (-1)^{|e_i| + |e_j| + n + n|e_j|} e^*_i \Big( \big\langle\!\big\langle \str( \underline\lambda' \At(d_{X\otimes X^*})^n (\psi(e_i)\otimes e^*_j )) \big\rangle\!\big\rangle e_j \Big) \nonumber \\
& = \big\langle\!\big\langle \str \big( \underline\lambda' \At(d_{X\otimes X^*})^n \big(\sum_i \psi(e_i)\otimes e^*_j \big)\big) \big\rangle\!\big\rangle \nonumber \\
& = \big\langle\!\big\langle \str \big( \underline\lambda' \At(d_{X\otimes X^*})^n (\psi)\big) \big\rangle\!\big\rangle \label{strlambdaAtn} \, .
\end{align}
To see that this is indeed equal to $\str(\psi)$ the idea is to convert the action of Atiyah classes on~$\psi$ to an action on $\underline\lambda'$ instead. More precisely, we will repeatedly use the super Jacobi identity
\be\label{superJacobi}
\big[ \varphi, \At(d_{X\otimes X^*}) \big] = (-1)^{|\varphi|} \big[ d_{X\otimes X^*}, \big[ \varphi, \nabla \big] \big] - \big[ \nabla, \big[ d_{X\otimes X^*}, \varphi \big] \big] 
\ee
for homogeneous $\varphi\in \End X$, together with the fact that the Atiyah class precomposed with the supertrace is zero. 

In the first step we apply~\eqref{superJacobi} with $\varphi=\underline\lambda'$ to see that~\eqref{strlambdaAtn} equals 
\be\label{2strterms}
\Big\langle\!\!\Big\langle \str\Big( \Big\{ (-1)^{n} \big[ d_{X\otimes X^*}, \big[ \underline\lambda', \nabla \big] \big] - \big[ \nabla, \big[ d_{X\otimes X^*}, \underline\lambda' \big] \big] \Big\} \big(\At(d_{X\otimes X^*})^{n-1}(\psi) \big) \Big) \Big\rangle\!\!\Big\rangle \, .
\ee
Since we defined $\psi=\psi'\underline\lambda\underline\mu$ with $[d_{X\otimes X^*}, \psi']=0$ we have $[d_{X\otimes X^*}, \psi]= \sum_i(\zeta_i f_i(x) + \xi_i g_i(z))$ for some maps~$\zeta_i$ and~$\xi_i$. Thus by cyclicity of $\str$ the first term in~\eqref{2strterms} is
\begin{align}
& \, \Big\langle\!\!\Big\langle \str\Big( \big[ \underline\lambda', \nabla \big] \big( \big[ d_{X\otimes X^*}, \At(d_{X\otimes X^*})^{n-1}(\psi) \big] \big) \Big) \Big\rangle\!\!\Big\rangle \nonumber \\
= & \, \Big\langle\!\!\Big\langle \str\Big( \big[ \underline\lambda', \nabla \big] \Big( \sum_i \At(d_{X\otimes X^*})^{n-1}\big( \zeta_i f_i(x) + \xi_i g_i(z)) \Big)\Big) \Big\rangle\!\!\Big\rangle \equiv \, 0 \, . \label{willdieinquotient} 
\end{align}
Here the last step follows immediately for the terms involving $g_i(z)=\partial_{z_i}V(z)$ as they are modded out by. The argument for the terms with $f_i(x) = \partial_{x_i}W(x)$ is slightly more subtle as they depend on the $x$-variables and not the $y$-variables featuring in the residue expression for the Kapustin-Li trace. However, the $f_i(x)$ in~\eqref{willdieinquotient} act as zero-forms multiplying $n$-forms (in the image under $(\varepsilon\circ\Psi)$ in $\langle\!\langle-\rangle\!\rangle$) from the right. Hence by the following lemma this is the same as scalar multiplication with $f_i(y)$, which does not survive the quotient. 

\begin{lemma}
Let $\omega\in\Omega(k[x_1,\ldots,x_n])$ be and $n$-form. Then $\Psi(\omega f)=\Psi(\omega) f(y)$ for any $f\in k[x]$. 
\end{lemma}

\begin{proof}
We may assume that $\omega$ is of the form $da_1\ldots da_n$ for some $a_i\in k[x]$. Then
\begin{align*}
\Psi(\omega f) & = \Psi \Big( \sum_{i=1}^n (-1)^{n-i} da_1\ldots da_{i-1} d(a_i a_{i+1}) da_{i+2}\ldots da_n df + (-1)^n a_1 da_2\ldots da_n df \Big) \\
& = \sum_{i=1}^n (-1)^{n-i} \partial_{[1]} a_1\ldots \partial_{[i-1]} a_{i-1} \partial_{[i]} (a_i a_{i+1}) \partial_{[i+1]} a_{i+2} \ldots \partial_{[n-1]} a_n \partial_{[n]} f \\
& \qquad + (-1)^n a_1 \partial_{[1]} a_2 \ldots \partial_{[n-1]} a_n \partial_{[n]} f \\
& = \sum_{i=1}^n (-1)^{n-i} \partial_{[1]} a_1\ldots \partial_{[i-1]} a_{i-1} \Big(\partial_{[i]} a_i {}	^{t_1\ldots t_i}a_{i+1} + {}^{t_1\ldots t_{i-1}} a_i \partial_{[i]} a_{i+1} \Big) \\
& \qquad \cdot  \partial_{[i+1]} a_{i+2} \ldots \partial_{[n-1]} a_n \partial_{[n]} f + (-1)^n a_1 \partial_{[1]} a_2 \ldots \partial_{[n-1]} a_n \partial_{[n]} f \\
& = \partial_{[1]} a_1 \ldots \partial_{[n]} a_n f(y) \\
& = \Psi(\omega) f(y) \, ,
\end{align*}
where we used lemma~\ref{TODO} in the third step. 
\end{proof}

Thus we find that only the second term in~\eqref{2strterms} remains. It is equal to 
\begin{align}
& \, -\Big\langle\!\!\Big\langle \str\Big( \nabla \Big( \big[ d_{X\otimes X^*}, \underline\lambda' \big] \big(\At(d_{X\otimes X^*})^{n-1}(\psi) \big) \Big)\Big) \Big\rangle\!\!\Big\rangle \nonumber \\
= & \, -\sum_{j=1}^n (-1)^{j+1} \Big\langle\!\!\Big\langle \nabla \Big\{ f_j \str\big( \underline\lambda'_j \At(d_{X\otimes X^*})^{n-1}(\psi) \big) \Big\} \Big\rangle\!\!\Big\rangle \label{sumfj}
\end{align}
where $\underline\lambda'_j := \lambda'_1\ldots \lambda'_{j-1} \widehat{\lambda'_j}\lambda'_{j+1}\ldots\lambda'_n$. Now we are in a similar situation as we were in~\eqref{strlambdaAtn}, only this time the relevant term $\str( \underline\lambda'_j \At(d_{X\otimes X^*})^{n-1}(\psi))$ has one Atiyah class less. Again we use $\str\circ \At=0$, apply the super Jacobi identity (now with $\varphi=\underline\lambda'_j$) and see that the first term on its right-hand side gives no contribution for the same reason as before, leading to
$$
\str(Z\psi) \equiv \sum_{j_2<j_1} \sum_{\sigma\in S_2} (-1)^{j_1+j_2 + |\sigma|} 
\Big\langle\!\!\Big\langle \nabla \Big\{ f_{j_{\sigma(1)}} \nabla \Big\{ f_{j_{\sigma(2)}} \str\big( \underline\lambda'_{j_{\sigma(1)} j_{\sigma(1)}} \At(d_{X\otimes X^*})^{n-2}(\psi) \big) \Big\} \Big\} \Big\rangle\!\!\Big\rangle \, .
$$
Continuing in this fasion we find that 
\begin{align}
\str(Z\psi) & \equiv \sum_{j_n<\ldots<j_1} \sum_{\sigma\in S_n} (-1)^{j_1+\ldots+j_n + |\sigma|} 
\Big\langle\!\!\Big\langle \nabla \Big\{ f_{j_{\sigma(1)}} \nabla \Big\{ f_{j_{\sigma(2)}} \nabla \Big\{ \ldots \nabla \Big\{ f_{j_{\sigma(n)}} \str\big( \psi \big) \Big\} \ldots \Big\}\Big\} \Big\} \Big\rangle\!\!\Big\rangle \nonumber \\
& \equiv (-1)^{n+1\choose 2} \sum_{\sigma\in S_n} (-1)^{|\sigma|} 
\Big\langle\!\!\Big\langle df_{\sigma(1)} df_{\sigma(2)} \ldots df_{\sigma(n)} \str(\psi) \Big\rangle\!\!\Big\rangle \nonumber \\
% Isn't it psi = psi(y) already at this next step, before taking the residue? 
& = \Res_{k[x']} \left[ \frac{(\varepsilon\circ\Psi) \str \left( \delta
\psi \right)}{f_1(x')\ldots f_n(x')} \right] \nonumber \\
& = \str(\psi)\Big|_{x'\longmapsto y} \nonumber
\end{align}
where 
$$
\delta = 
\det 
\begin{pmatrix}
df_1 & df_2 & \cdots & df_n \\
\vdots & \vdots & & \vdots \\
df_1 & df_2 & \cdots & df_n 
\end{pmatrix} .
$$
This concludes the proof that the Zorro map~\eqref{Zorro1superdetail} is homotopic to the identity. 

The three other Zorro moves are proven analogously. What remains to be done is to show that the same is true if we replace the completed bar complex~$\cBar$ by the unit object~$\Delta_W$ of the monoidal category $\HMF(\Ae, \widetilde W)$. 

[TODO: better write this bit only once the section on $\cBar$ as a MF is in place] 

\begin{theorem}
TODO: LG has dualities\ldots
\end{theorem}


\section{Defect action on bulk fields}\label{sec:defectaction}

In any bicategory with duals there are natural maps between the endomorphism spaces of unit 1-cells. Roughly, these maps are constructed by capturing a 2-morphism of a unit 1-cell inside a loop labelled by an arbitrary 1-cell (and its dual). Below we present the details for the case of the bicategory $\mathcal{LG}$. We will also give the interpretation in terms of defect actions on bulk fields in Landau-Ginzburg models. 

Let $X\in \hmf(k[z,x], V-W)$ as before. We define a map 
$$
\mathcal D_X: \End(\Delta_V) \longrightarrow \End(\Delta_W) 
$$
in terms of the morphisms encoding the monoidal and duality structures as follows. For $\phi\in \End(\Delta_V)$ we set $\mathcal D_X(\phi) = \eval_X \circ (1_{X^*}\otimes (\lambda_X \circ (\phi\otimes 1_X)\circ \lambda_X^{-1})) \circ \widetilde\coev_X$. Diagrammatically this definition reads
\be\label{defectaction}
\mathcal D_X(\phi) = 
\begin{tikzpicture}[very thick,scale=0.8,color=blue!50!black, baseline]

\fill (1.5,1.3) circle (2.5pt) node[right] {{\small $\lambda_X$}};
\fill (1.5,-1.3) circle (2.5pt) node[right] {{\small $\lambda^{-1}_X$}};
\fill (0,0) circle (2.5pt) node[left] {{\small $\phi$}};

\fill (0.6,1) circle (0pt) node {{\small $\Delta_V$}};
\fill (0.6,-0.95) circle (0pt) node {{\small $\Delta_V$}};

\draw[directed] (1.5,1.3) .. controls +(0,1.5) and +(0,1.5) .. (-1.5,1.3);
\draw[directed] (-1.5,-1.3) .. controls +(0,-1.5) and +(0,-1.5) .. (1.5,-1.3);
\draw (1.5,-1.3) -- (1.5,1.3)
node[midway,left] {{{\footnotesize$z'\vphantom{y}$}}}
node[midway,right] {{{\footnotesize$y\vphantom{yz'}$}}};
\draw (-1.5,-1.3) -- (-1.5,1.3)
node[midway,left] {{{\footnotesize$x\vphantom{yz'}$}}}
node[midway,right] {{{\footnotesize$z\vphantom{yz'}$}}};
\draw[dashed] (0,0) .. controls +(0,1) and +(-0.5,-1) .. (1.5,1.3);
\draw[dashed] (0,0) .. controls +(0,-1) and +(-0.5,1) .. (1.5,-1.3);
\draw[dashed] (0,-2.5) -- (0,-3.5)
node[near end,right] {{{\small$\Delta_W$}}};
\draw[dashed] (0,2.47) -- (0,3.5)
node[near end,right] {{{\small$\Delta_W$}}};
\end{tikzpicture}
\ee
where again we indicated our choice of variable names in the four domains. 

\begin{remark}
$\End(\Delta_W) = k[x]/(\partial_{x_i}W)$ is the Hochschild cohomology of $\hmf(k[x], W)$~\cite{d0904.4713}. This space also precisely describes bulk fields of Landau-Ginzburg models with potential~$W$. Furthermore, matrix factorisations of $V-W$ describe defect conditions between different Landau-Ginzburg models. Hence the map~\eqref{defectaction} has the natural interpretation of defect operators on bulk fields: a bulk field~$\phi$ in the theory with potential~$V$ is mapped to the bulk field $\mathcal D_X(\phi)$ in the theory with potential~$W$ by wrapping around its insertion on the worldsheet a defect line labelled by~$X$, and then collapsing this loop onto the insertion point. This limiting process is non-singular as the bicategory $\mathcal{LG}$ describes the purely topological sector of Landau-Ginzburg models. 
\end{remark}

Using the ``folding trick'' (which relates defects to boundary conditions in a product theory) one can argue for an explicit expression for $\mathcal D_X(\phi)$. This was done in~\cite{cr1006.5609} for the case $V=W$. Here we use the duality structure to directly prove it for the general case: 

\begin{proposition}
For any $X\in \hmf(k[z,x], V-W)$ and $\phi\in \End(\Delta_V)$ we have
$$
\mathcal D_X(\phi) = (-1)^n \Res_{k[z]} \left[ \frac{\phi(z) \str\big( \partial_{x_1} d_{X^*}\ldots \partial_{x_n} d_{X^*} \partial_{z_1} d_{X^*}\ldots \partial_{z_m} d_{X^*} \big)}{\partial_{z_1} V \ldots \partial_{z_m} V} \right] . 
$$
\end{proposition}

\begin{proof}
Since $\End(\Delta_W) = k[x]/(\partial_{x_i}W)$ and $\End(\Delta_V) = k[z]/(\partial_{z_i}V)$ we are free to set $x=y$ and $z=z'$ at appropriate places, cf.~\eqref{defectaction}. Furthermore, $\lambda_X$ will project out all non-zero degree contributions coming from the action of $\lambda_X^{-1}$, so $\lambda_X\circ (\phi \otimes 1_X)\circ \lambda_X^{-1}$ is simply multiplication by the polynomial $\phi(z)$. 

In the lower part of~\eqref{defectaction} we have 
\begin{align}
\widetilde\coev (1) & = \sum_j (-1)^{|e_j|} (\varepsilon\Psi) \left( (-\At(d_{X^*}))^n (e_j^*\otimes e_j) \right) \nonumber \\
& = \sum_j (-1)^{|e_j| + n} (-1)^{(|e_j| + n)+\ldots +(|e_j| + n) + n|e_j|} e^*_{l_n} \otimes e_j \otimes (\varepsilon\Psi) \left( d(d_{X^*})_{l_n l_{n-1}} \ldots d(d_{X^*})_{l_1 j} \right) \nonumber \\
& = \sum_j (-1)^{|e_j| + n + {n+1\choose 2}} \big( \partial_{[1]} d_{X^*} \ldots \partial_{[n]} d_{X^*} (e^*_j) \big) \otimes e_j \nonumber \\
& =  (-1)^{n + {n+1\choose 2}} \Big( \big( \partial_{x_1} d_{X^*} \ldots \partial_{x_n} d_{X^*} \big) \otimes 1_X \Big) \circ \Big( \sum_j (-1)^{|e_j|} e^*_j \otimes e_j \Big) \label{coevtilde1}
\end{align}
which we identify with $(-1)^{n + {n+1\choose 2}} \partial_{x_1} d_{X^*} \ldots \partial_{x_n} d_{X^*}$ in $\End(X^*)$. Note that in the last step leading to~\eqref{coevtilde1} we set $\partial_{[i]} d_{X^*}(x,z) = \partial_{x_i} d_{X^*}(x,z)$ since $x=y$ in $\End(\Delta_W)$. 

Next we apply the upper part of~\eqref{defectaction} to~\eqref{coevtilde1} to get
$$
\mathcal D_X(\phi) = (-1)^n \Res_{k[z]} \left[ \frac{\phi(z) \str\big( \partial_{x_1} d_{X^*}\ldots \partial_{x_n} d_{X^*} \partial_{z_1} d_{X^*}\ldots \partial_{z_m} d_{X^*} \big)}{\partial_{z_1} V \ldots \partial_{z_m} V} \right] + \mathcal O(\theta) \, . 
$$
Here we collectively denote the contributions from $\eval_X$ of non-zero degree in the Koszul complex $\Delta_W$ by $\mathcal O(\theta)$. Since we know that $\mathcal D_X(\phi)$ is a morphism in $\End_{\hmf(k[x,y], \widetilde W)}(\Delta_W) = k[x]/(\partial_{x_i})$ it follows that $\mathcal O(\theta)$ must be null-homotopic, thus concluding the proof. 
\end{proof}





\newcommand{\etalchar}[1]{$^{#1}$}
\providecommand{\href}[2]{#2}
\begin{thebibliography}{FYH{\etalchar{+}}85}

%\bibitem[AS]{as1105.5117}
%M.~Aganagic and S.~Shakirov, \emph{Knot {H}omology from {R}efined
%  {C}hern-{S}imons {T}heory},
%  \href{http://arxiv.org/abs/1105.5117}{[arXiv:1105.5117]}.
%
%\bibitem[BN]{bnKhovanov11crossings}
%D.~Bar-Natan, \emph{Khovanov {H}omology for {K}nots and {L}inks with up to 11
%  {C}rossings}, available at
%  \href{http://www.math.toronto.edu/drorbn/papers/KHTables/KHTables.pdf}{http:%
%//www.math.toronto.edu/drorbn/papers/KHTables/KHTables.pdf}.
%
%\bibitem[Bec]{b1105.0702}
%H.~Becker, \emph{Khovanov-Rozansky homology via Cohen-Macaulay approximations and Soergel bimodules},
%  \href{http://arxiv.org/abs/1105.0702}{[arXiv:1105.0702]}.
%
%\bibitem[BR07]{br0707.0922}
%I.~Brunner and D.~Roggenkamp, \emph{B-type defects in {L}andau-{G}inzburg
%  models}, JHEP \textbf{0708} (2007), 093,
%  \href{http://arxiv.org/abs/0707.0922}{[arXiv:0707.0922]}.
%
%\bibitem[CF94]{cf9405183}
%L.~Crane and I.~B. Frenkel, \emph{Four dimensional topological quantum field
%  theory, {H}opf categories, and the canonical bases}, J. Math. Phys.
%  \textbf{35} (1994), 5136--5154,
%  \href{http://arxiv.org/abs/hep-th/9405183}{[hep-th/9405183]}.
%
%\bibitem[CK08a]{ck0701194}
%S.~Cautis and J.~Kamnitzer, \emph{Knot homology via derived categories of
%  coherent sheaves {I}, {$sl(2)$} case}, Duke Math. J. \textbf{142} (2008),
%  511--588, \href{http://arxiv.org/abs/math/0701194}{[math.AG/0701194]}.
%
%\bibitem[CK08b]{ck0710.3216}
%S.~Cautis and J.~Kamnitzer, \emph{Knot homology via derived categories of coherent sheaves {II},
%  {$sl(m)$} case}, Invent. Math. \textbf{174} (2008), 165--232,
%  \href{http://arxiv.org/abs/math/0710.3216}{[math.AG/0710.3216]}.
%
%\bibitem[CM]{cmWebCompileCode}
%N.~Carqueville and D.~Murfet, \emph{Code to compute {K}hovanov-{R}ozansky
%  homology and defect fusion in {L}andau-{G}inzburg models},
%  \href{http://www.carqueville.net/nils/webCompilations}{http://www.carqueville.net/nils/webCompilations}.

\bibitem[CR]{cr1006.5609}
N.~Carqueville and I.~Runkel, \emph{Rigidity and defect actions in
  Landau-Ginzburg models}, Comm. Math. Phys. \textbf{310} (2012) 135--179, 
  \href{http://arxiv.org/abs/1006.5609}{[arXiv:1006.5609]}.

%\bibitem[CR10]{cr0909.4381}
%N.~Carqueville and I.~Runkel, \emph{On the monoidal structure of matrix bi-factorisations}, J. Phys.
%  A: Math. Theor. \textbf{43} (2010), 275401,
%  \href{http://arxiv.org/abs/0909.4381}{[arXiv:0909.4381]}.
%
%\bibitem[Cra]{c0403266}
%M.~Crainic, \emph{On the perturbation lemma, and deformations},
%  \href{http://arxiv.org/abs/math/0403266}{[math.AT/0403266]}.
%
%\bibitem[DGR06]{dgr0505662}
%N.~M. Dunfield, S.~Gukov, and J.~Rasmussen, \emph{The {S}uperpolynomial for
%  {K}not {H}omologies}, Experimental Math. \textbf{15} (2006), 129--159,
%  \href{http://arxiv.org/abs/math/0505662}{[math.GT/0505662]}.
%
%\bibitem[DKR]{dkr1107.0495}
%A.~Davydov, L.~Kong, and I.~Runkel, \emph{Field theories with defects and the
%  centre functor}, \href{http://arxiv.org/abs/1107.0495}{[arXiv:1107.0495]}.
%
%\bibitem[DBM{\etalchar{+}}11]{dbmmss1106.4305}
%P.~Dunin-Barkowski, A.~Mironov, A.~Morozov, A.~Sleptsov, A.~Smirnov, \emph{Superpolynomials for toric knots from evolution induced by cut-and-join operators},
%  \href{http://arxiv.org/abs/1106.4305}{[arXiv:1106.4305]}. 
%  
\bibitem[Dyc]{d0904.4713}
T.~Dyckerhoff, \emph{Compact generators in categories of matrix factorizations},
  Duke Math. J. \textbf{159} (2011), 223--274,
  \href{http://arxiv.org/abs/0904.4713}{[arXiv:0904.4713]}.

%\bibitem[DM]{dm1102.2957}
%T.~Dyckerhoff and D.~Murfet, \emph{Pushing forward matrix factorisations},
%  \href{http://arxiv.org/abs/1102.2957}{[arXiv:1102.2957]}.
%
%\bibitem[FFRS07]{ffrs0607247}
%J.~Fr\"ohlich, J.~Fuchs, I.~Runkel, and C.~Schweigert, \emph{Duality and
%  defects in rational conformal field theory}, Nucl. Phys. B \textbf{763}
%  (2007), 354--430,
%  \href{http://arxiv.org/abs/hep-th/0607247}{[hep-th/0607247]}.
%
%\bibitem[FYH{\etalchar{+}}85]{Homfly}
%P.~Freyd, D.~Yetter, J.~Hoste, W.~B.~R. Lickorish, K.~Millett, and A.~Ocneanu,
%  \emph{A new polynomial invariant of knots and links}, Bull. Amer. Math. Soc.
%  \textbf{12} (1985), 239--246.
%
%\bibitem[GIKV10]{gikv0705.1368}
%S.~Gukov, A.~Iqbal, C.~Koz\c{c}az, and C.~Vafa, \emph{Link {H}omologies and the
%  {R}efined {T}opological {V}ertex}, Comm. Math. Phys. \textbf{298} (2010),
%  757--785, \href{http://arxiv.org/abs/0705.1368}{[arXiv:0705.1368]}.
%
%\bibitem[GSV05]{gsv0412243}
%S.~Gukov, A.~Schwarz, and C.~Vafa, \emph{Khovanov-{R}ozansky {H}omology and
%  {T}opological {S}trings}, Lett. Math. Phys. \textbf{74} (2005), 53--74,
%  \href{http://arxiv.org/abs/hep-th/0412243}{[hep-th/0412243]}.
%
%\bibitem[GV99]{gv9811131}
%R.~Gopakumar and C.~Vafa, \emph{On the {G}auge {T}heory/{G}eometry
%  {C}orrespondence}, Adv. Theor. Math. Phys. \textbf{3} (1999), 1415--1443,
%  \href{http://arxiv.org/abs/hep-th/9811131}{[hep-th/9811131]}.
%
%\bibitem[GW]{gw0512298}
%S.~Gukov and J.~Walcher, \emph{Matrix {F}actorizations and {K}auffman
%  {H}omology}, \href{http://arxiv.org/abs/hep-th/0512298}{[hep-th/0512298]}.
%
%\bibitem[Jae]{j1101.3302}
%T.~C. Jaeger, \emph{Khovanov-{R}ozansky {H}omology and {C}onway {M}utation},
%  \href{http://arxiv.org/abs/1101.3302}{[arXiv:1101.3302]}.
%
%\bibitem[Jon85]{JonesPolynomialPaper}
%V.~F.~R. Jones, \emph{A polynomial invariant for knots via von {N}eumann
%  algebras}, Bull. Amer. Math. Soc. \textbf{12} (1985), 103--111.
%
%\bibitem[Kap]{k1004.2307}
%A.~Kapustin, \emph{Topological {F}ield {T}heory, {H}igher {C}ategories, and
%  {T}heir {A}pplications},
%  \href{http://arxiv.org/abs/1004.2307}{[arXiv:1004.2307]}.
%
%\bibitem[Kaw96]{kawauchibook}
%A.~Kawauchi, \emph{A {S}urvey of {K}not {T}heory}, Birkh\"auser, 1996.
%
%\bibitem[Kho00]{k9908171}
%M.~Khovanov, \emph{A categorification of the {J}ones polynomial}, Duke Math. J.
%  \textbf{101} (2000), 359--426,
%  \href{http://arxiv.org/abs/math/9908171}{[math.QA/9908171]}.
%
%\bibitem[Kho07]{k0510265}
%M.~Khovanov, \emph{Triply-graded link homology and Hochschild homology of Soergel bimodules},
%  Int. Journal of Math. \textbf{18} (2007), 869--885,
%  \href{http://arxiv.org/abs/math/0510265}{[math.GT/0510265]}.
%
%\bibitem[KR07a]{kr0701333}
%M.~Khovanov and L.~Rozansky, \emph{Virtual crossings, convolutions and a
%  categorification of the {$\operatorname{SO}(2N)$} {K}auffman polynomial},
%  Journal of G\"okova Geometry Topology \textbf{1} (2007), 116--214,
%  \href{http://arxiv.org/abs/math/0701333}{[math.QA/0701333]}.
%
%\bibitem[KR07b]{kr0404189}
%M.~Khovanov and L.~Rozansky, \emph{Topological Landau-Ginzburg models on the world-sheet foam},
%  Adv. Theor. Math. Phys. \textbf{11} (2007), 233--259,
%  \href{http://arxiv.org/abs/hep-th/0404189}{[hep-th/0404189]}.
%
%\bibitem[KR08a]{kr0401268}
%M.~Khovanov and L.~Rozansky, \emph{Matrix factorizations and link homology}, Fund. Math.
%  \textbf{199} (2008), 1--91,
%  \href{http://arxiv.org/abs/math/0401268}{[math/0401268]}.
%
%\bibitem[KR08b]{kr0505056}
%M.~Khovanov and L.~Rozansky, \emph{Matrix factorizations and link homology {II}}, Geometry \&
%  Topology \textbf{12} (2008), 1387--1425,
%  \href{http://arxiv.org/abs/math/0505056}{[math.QA/0505056]}.
%
%\bibitem[Lam86]{LambekRingsModules}
%J.~Lambek, \emph{Lectures on rings and modules}, AMS Chelsea Publishing, 1986.
%
%\bibitem[LM]{Calinetal2}
%C.~I. Lazaroiu and D.~McNamee, unpublished.
%
%\bibitem[LMnV00]{lmv0010102}
%J.~M.~F. Labastida, M.~Mari\~{n}o, and C.~Vafa, \emph{Knot {I}nvariants and
%  {T}opological {S}trings}, JHEP \textbf{0011} (2000), 007,
%  \href{http://arxiv.org/abs/hep-th/0010102}{[hep-th/0010102]}.
%
%\bibitem[MSV09]{msv0708.2228}
%M.~Mackaay, M.~Sto\v{s}i\'{c}, and P.~Vaz, \emph{$\mathfrak{sl}(N)$ link homology ($N\geq 4$) using foams and the Kapustin-Li formula}, Geometry \& Topology \textbf{13} (2009), 1075--1128,
%  \href{http://arxiv.org/abs/0708.2228}{[arXiv:0708.2228]}.
%
%\bibitem[Man07]{m0601629}
%C.~Manolescu, \emph{Link homology theories from symplectic geometry}, Adv. in
%  Math. \textbf{211} (2007), 363--416,
%  \href{http://www.arxiv.org/abs/math.AG/0601629}{[math.SG/0601629]}.
%
%\bibitem[McN09]{McNameethesis}
%D.~McNamee, \emph{On the mathematical structure of topological defects in
%  {L}andau-{G}inzburg models}, MSc Thesis, Trinity College Dublin, 2009.
%
%\bibitem[Mn05]{marinoknotbook}
%M.~Mari\~{n}o, \emph{Chern-{S}imons {T}heory, {M}atrix {M}odels, and
%  {T}opological s{t}rings}, Oxford University Press, 2005.
%
%\bibitem[MOY98]{moy1998}
%H.~Murakami, T.~Ohtsuki, and S.~Yamada, \emph{Homfly polynomial via an
%  invariant of colored plane graphs}, Enseign. Math. \textbf{44} (1998),
%  325--360.
%
%\bibitem[MS]{ms0709.1971}
%V.~Manzorchuck and C.~Stroppel, \emph{A combinatorial approach to functorial
%  quantum {$\mathfrak{sl}_k$} knot invariants},
%  \href{http://arxiv.org/abs/0709.1971}{[arXiv:0709.1971]}.
%
%\bibitem[OV00]{ov9912123}
%H.~Ooguri and C.~Vafa, \emph{Knot {I}nvariants and {T}opological {S}trings},
%  Nucl. Phys. B \textbf{577} (2000), 419--438,
%  \href{http://arxiv.org/abs/hep-th/9912123}{[hep-th/9912123]}.
%
%\bibitem[PT87]{HomflyPT}
%J.~Przytycki and P.~Traczyk, \emph{Conway algebras and skein equivalence of
%  links}, Proc. Amer. Math. Soc. \textbf{100} (1987), 744--748.
%
%\bibitem[Ras]{r0607544}
%J.~Rasmussen, \emph{Some differentials on {K}hovanov-{R}ozansky homology},
%  \href{http://arxiv.org/abs/math/0607544}{[math.GT/0607544]}.
%
%\bibitem[Ras07]{r0508510}
%J.~Rasmussen, \emph{Khovanov-{R}ozansky homology of two-bridge knots and links},
%  Duke Math. J. \textbf{136} (2007), 551--583,
%  \href{http://arxiv.org/abs/math/0508510}{[math.GT/0508510]}.
%  
%\bibitem[Ric94]{rickard}
%J.~Rickard, \emph{Translation functors and equivalences of derived categories for blocks of algebraic groups}, in “Finite dimensional algebras and related topics”, Kluwer (1994), 255-–264.
%
%\bibitem[Rou06]{RouquierMexico}
%R.~Rouquier, \emph{Categorification of {$\mathfrak{sl}_{2}$} and braid groups},
%  Trends in representation theory of algebras and related topics (2006),
%  137--167.
%
%\bibitem[RT90]{RT1990}
%N.~Reshetikhin and V.~Turaev, \emph{Ribbon graphs and their invariants derived
%  from quantum groups}, Comm. Math. Phys. \textbf{127} (190), 1--26.
%
%\bibitem[RT91]{RT1991}
%N.~Reshetikhin and V.~Turaev, \emph{Invariants of 3-manifolds via link polynomials and quantum
%  groups}, Invent. Math. \textbf{103} (1991), 547--597.
%
%\bibitem[SS06]{ss0405089}
%P.~Seidel and I.~Smith, \emph{A link invariant from the symplectic geometry of
%  nilpotent slices}, Duke Math. J. \textbf{134} (2006), 453--514,
%  \href{http://arxiv.org/abs/math/0405089}{[math.SG/0405089]}.
%
%\bibitem[Str05]{sCatTLcTCpf}
%C.~Stroppel, \emph{Categorification of the {T}emperley-{L}ieb category,
%  tangles, and cobordisms via projective functors}, Duke Math. J. \textbf{126}
%  (2005), 547--596.
%
%\bibitem[Sus]{s0701045}
%J.~Sussan, \emph{Category {$\mathcal O$} and {$\mathfrak{sl}_k$} link
%  invariants}, \href{http://arxiv.org/abs/math/0701045}{[math.QA/0701045]}.
%
%\bibitem[Tur88]{t1988YB}
%V.~Turaev, \emph{The {Y}ang-{B}axter equation and invariants of links}, Invent.
%  Math. \textbf{92} (1988), 527--553.
%
%\bibitem[Tur10]{turaevbook}
%V.~Turaev, \emph{Quantum invariants of knots and 3-manifolds}, de Gruyter, 2010,
%  2nd edition.
%
%\bibitem[Weba]{w0610650}
%B.~Webster, \emph{Khovanov-Rozansky homology via a canopolis formalism},
%  \href{http://arxiv.org/abs/math/0610650}{[math.GT/0610650]}.
%
%\bibitem[Webb]{w1005.4559}
%B.~Webster, \emph{Knot invariants and higher representation theory {II}: the
%  categorification of quantum knot invariants},
%  \href{http://arxiv.org/abs/1005.4559}{[arXiv:1005.4559]}.
%
%\bibitem[Wit]{w1101.3216}
%E.~Witten, \emph{Fivebranes and {K}nots},
%  \href{http://arxiv.org/abs/1101.3216}{[arXiv:1101.3216]}.
%
%\bibitem[Wit89]{wittenjones}
%E.~Witten, \emph{Quantum field theory and the {J}ones polynomial}, Comm. Math.
%  Phys. \textbf{121} (1989), 351--399.
%
%\bibitem[Wit95]{w9207094}
%E.~Witten, \emph{Chern-{S}imons {G}auge {T}heory {A}s {A} {S}tring {T}heory},
%  Prog. Math. \textbf{133} (1995), 637--678,
%  \href{http://arxiv.org/abs/hep-th/9207094}{[hep-th/9207094]}.
%
%\bibitem[Wu]{w0907.0695}
%H.~Wu, \emph{A colored {$\mathfrak{sl}(N)$}-homology for links in {$S^3$}},
%  \href{http://arxiv.org/abs/0907.0695}{[arXiv:0907.0695]}.
%
%\bibitem[Wu08]{w0508064}
%H.~Wu, \emph{Braids, {T}ransversal links and the {K}hovanov-{R}ozansky {T}heory},
%  \href{http://arxiv.org/abs/math/0508064}{[math.GT/0508064]}, Trans. Amer. Math. Soc. \textbf{360}
%  (2008), 3365--3389.
%
%\bibitem[Yon]{y0906.0220}
%Y.~Yonezawa, \emph{Quantum {$(\mathfrak{sl}_n, \wedge V_n)$} link invariant and
%  matrix factorizations},
%  \href{http://arxiv.org/abs/0906.0220}{[arXiv:0906.0220]}.

\end{thebibliography}

\end{document}
